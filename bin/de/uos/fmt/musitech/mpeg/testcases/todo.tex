 \section*{1 CWSMR}
 \subsection*{1.1 coding classic form}
 \begin{tabular}{|p{.5\textwidth}|p{.5\textwidth}|}
 \hline
 \textbf{TODO} &
 \textbf{Solution} 
\\      \hline
 Naming of Staffs &
  
\\ 	\hline
 move rests away vertically &
  
\\ 	\hline
 Staff brace &
  done. (quite ugly though) 
\\ 	\hline
 Measure numbers (are these the rehearsal marks from the .doc?) &
  
\\ 	\hline
 (f) &
  done 
\\ 	\hline
 Small notes &
  A NotationChord now may have another NotationChord as a child. This child is called entryChord. This recursion is necessary as the small notes in the example are not just small notes, but are part of the following note, meaning that they don't have a metrical duration of their own. This problem is solved graphically in the same manner: a chord has an entryChord as child with extends the width of the parent-chord. 
\\ 	\hline
 Dotted and paranthised notes and slurs &
  
\\ 	\hline
 a part which is the same in two voices is rendered only once and ''a 2'' is written above the staff &
  The representation contains both parts equally. One part has the rendering hint ''duplicate of other voice'' which results in all notes getting invisible and the string being generated. 
\\ 	\hline
 the ''a 2'' string has to be added for the duplicate term in the oboe voice, 3rd measure. &
  for now this is done as a simple StringSymbol on a MetricAttachable 
\\ 	\hline
 What to do with two voices rendered as one with chords (bassoon staff). Are these then chords in the representation or are they different voices with rendering hints? &
  Yes. Two times the same voices with rendering hint ''voices as chords'' on the NotationStaff 
\\ 	\hline
 Staccato-Accents &
  This is done in a simple way. But probably this is unsatisfying. The crux is that there may be notes where a staccato point belongs (or even worse beams). It will be hard to check that and even harder to find free spot. 
\\ 	\hline
 fix weird alignment bug (third voice) &
  done 
\\ 	\hline
 should NotationStaff.setTransposition() be semantical? &
  Yes it should. done. NotationStaff.setTransposition() takes a value between -12 and 12 and transposes that many halftones. (smaller or greater values should also be possible, but have no graphical effect (should they?)) 
\\ 	\hline
 scaling for some staffs &
  scaling of gui.Staff seems to work. As this is the first version, there'll be bugs later on 
\\ 	\hline
 lyrics &
  
\\ 	\hline
 voice and staff distinction has to be rethought (voice changing staffs midway) &
  The voices don't change staffs. They just overlap inside one staff... This is already possible. done 
\\ 	\hline
 Measure numbers &
  
\\ 	\hline
 long rests &
  done 
\\ 	\hline
 Stacked objects attached to notes &
  done. see class data.score.StackSymbol on the representation side and score.gui.CustomStackScoreObject on the rendering side 
\\ 	\hline
 triols including rests &
  done. although there're some quirks regarding beam points 
\\ 	\hline
 trills &
  
\\ 	\hline
 indented staffs &
  
\\ 	\hline
 accents above staff &
  
\\ 	\hline
 \end{tabular} \subsection*{1.2 coding romantic form}
 \begin{tabular}{|p{.5\textwidth}|p{.5\textwidth}|}
 \hline
 \textbf{TODO} &
 \textbf{Solution} 
\\
 \hline
	aufloesungszeichen before \# &
  
\\ 	\hline
	fingerings &
  
\\ 	\hline
	complex slurs &
  
\\ 	\hline
	pedal marks &
  
\\ 	\hline
 &
  
\\ 	\hline
	extension lines &
  
\\ 	\hline
	volume hints &
  
\\ 	\hline
	beams with rests &
  
\\ 	\hline
	notes with two stems &
  
\\ 	\hline
	notes belonging to two beams &
  
\\ 	\hline
	beams crossing multiple measures &
  
\\ 	\hline
	one note being small and large &
  
\\ 	\hline
	weird spanners &
  
\\ 	\hline
	beams across staffs &
  
\\ 	\hline
	paranthised accidentals &
  
\\ 	\hline
	vertically broken barline &
  
\\ 	\hline
	cluster notation &
  
\\ 	\hline
 \end{tabular} \subsection*{1.3 coding jazz}
 \begin{tabular}{|p{.5\textwidth}|p{.5\textwidth}|}
 \hline
 \textbf{TODO} &
 \textbf{Solution} 
\\
 \hline
	rhythm section &
  
\\ 	\hline
	using jazz font &
  
\\ 	\hline
 &
  
\\ 	\hline
 \end{tabular} \subsection*{1.4 coding Pop/Rock}
 \begin{tabular}{|p{.5\textwidth}|p{.5\textwidth}|}
 \hline
 \textbf{TODO} &
 \textbf{Solution} 
\\
 \hline
	tablature &
  
\\ 	\hline
	chord names &
  
\\ 	\hline
	lyrics with word extenders &
  
\\ 	\hline
	repeat sign &
  
\\ 	\hline
	slash notation &
  
\\ 	\hline
 \end{tabular} \subsection*{1.5 modeling of symbolic events}
 \begin{tabular}{|p{.5\textwidth}|p{.5\textwidth}|}
 \hline
 \textbf{TODO} &
 \textbf{Solution} 
\\
 \hline
 \end{tabular} \subsection*{1.6 duration of symbolic events}
 \begin{tabular}{|p{.5\textwidth}|p{.5\textwidth}|}
 \hline
 \textbf{TODO} &
 \textbf{Solution} 
\\
 \hline
	nested tuplets &
  
\\ 	\hline
	???????? &
  
\\ 	\hline
 \end{tabular} \subsection*{1.7 ordering relationships of symbolic events}
 \begin{tabular}{|p{.5\textwidth}|p{.5\textwidth}|}
 \hline
 \textbf{TODO} &
 \textbf{Solution} 
\\
 \hline
	weird alignment &
  
\\ 	\hline
	stem crossing staves &
  
\\ 	\hline
 \end{tabular} \subsection*{1.8 symbolic qualifiers}
 \begin{tabular}{|p{.5\textwidth}|p{.5\textwidth}|}
 \hline
 \textbf{TODO} &
 \textbf{Solution} 
\\
 \hline
	???????? &
  
\\ 	\hline
 \end{tabular} \subsection*{1.9 symbolic selections}
 \begin{tabular}{|p{.5\textwidth}|p{.5\textwidth}|}
 \hline
 \textbf{TODO} &
 \textbf{Solution} 
\\
 \hline
	implement selection &
  
\\ 	\hline
 \end{tabular} \section*{2 different codings}
 \subsection*{2.1 tablatures}
 \begin{tabular}{|p{.5\textwidth}|p{.5\textwidth}|}
 \hline
 \textbf{TODO} &
 \textbf{Solution} 
\\
 \hline
	tablatures &
  
\\ 	\hline
 \end{tabular} \subsection*{2.2 braille}
 \begin{tabular}{|p{.5\textwidth}|p{.5\textwidth}|}
 \hline
 \textbf{TODO} &
 \textbf{Solution} 
\\
 \hline
	braille &
  
\\ 	\hline
	braille with different rendering hints &
  
\\ 	\hline
 \end{tabular} \subsection*{2.3 spoken music}
 \begin{tabular}{|p{.5\textwidth}|p{.5\textwidth}|}
 \hline
 \textbf{TODO} &
 \textbf{Solution} 
\\
 \hline
	spoken music &
  
\\ 	\hline
 \end{tabular} \subsection*{2.4 percussions}
 \begin{tabular}{|p{.5\textwidth}|p{.5\textwidth}|}
 \hline
 \textbf{TODO} &
 \textbf{Solution} 
\\
 \hline
	new note heads &
  
\\ 	\hline
	voice/instrument is determined by note height &
  
\\ 	\hline
 \end{tabular} \subsection*{2.5 baroque}
 \begin{tabular}{|p{.5\textwidth}|p{.5\textwidth}|}
 \hline
 \textbf{TODO} &
 \textbf{Solution} 
\\
 \hline
	weird alignment of 1/32 and 1/16 &
  
\\ 	\hline
 \end{tabular} \subsection*{2.6 20th century experimental}
 \begin{tabular}{|p{.5\textwidth}|p{.5\textwidth}|}
 \hline
 \textbf{TODO} &
 \textbf{Solution} 
\\
 \hline
	holding marks for notes &
  
\\ 	\hline
	gradually increasing beams &
  
\\ 	\hline
	everything &
  
\\ 	\hline
 \end{tabular} \subsection*{2.7 schenkerian analysis}
 \begin{tabular}{|p{.5\textwidth}|p{.5\textwidth}|}
 \hline
 \textbf{TODO} &
 \textbf{Solution} 
\\
 \hline
	curves of different varities in addition to the score &
  
\\ 	\hline
 \end{tabular} \subsection*{2.8 neumes}
 \begin{tabular}{|p{.5\textwidth}|p{.5\textwidth}|}
 \hline
 \textbf{TODO} &
 \textbf{Solution} 
\\
 \hline
	different font &
  
\\ 	\hline
 \end{tabular} \section*{3 user defined symbolic elements}
 \subsection*{3.1 symbolic events}
 \begin{tabular}{|p{.5\textwidth}|p{.5\textwidth}|}
 \hline
 \textbf{TODO} &
 \textbf{Solution} 
\\
 \hline
	generic ScoreObject &
  
\\ 	\hline
 \end{tabular} \subsection*{3.2 symbolic qualifiers}
 \begin{tabular}{|p{.5\textwidth}|p{.5\textwidth}|}
 \hline
 \textbf{TODO} &
 \textbf{Solution} 
\\
 \hline
	generic ScoreObject which can be bound to another? &
  
\\ 	\hline
 \end{tabular} \subsection*{3.3 SMR Context}
 \begin{tabular}{|p{.5\textwidth}|p{.5\textwidth}|}
 \hline
 \textbf{TODO} &
 \textbf{Solution} 
\\
 \hline
	??? &
  
\\ 	\hline
 \end{tabular} \subsection*{3.4 symbolic selections}
 \begin{tabular}{|p{.5\textwidth}|p{.5\textwidth}|}
 \hline
 \textbf{TODO} &
 \textbf{Solution} 
\\
 \hline
 &
  
\\ 	\hline
 \end{tabular} \section*{4 audio and video rendering}
 \subsection*{4.1 audio rendering}
 \begin{tabular}{|p{.5\textwidth}|p{.5\textwidth}|}
 \hline
 \textbf{TODO} &
 \textbf{Solution} 
\\
 \hline
	this is the same as one test case in req. 1 &
  
\\ 	\hline
 \end{tabular} \subsection*{4.2 visual rendering}
 \begin{tabular}{|p{.5\textwidth}|p{.5\textwidth}|}
 \hline
 \textbf{TODO} &
 \textbf{Solution} 
\\
 \hline
	this is the same as one test case in req. 1 &
  
\\ 	\hline
 \end{tabular} \section*{5 linking of symbolic elements}
 \subsection*{5.1 Symbolic Qualifiers}
 \begin{tabular}{|p{.5\textwidth}|p{.5\textwidth}|}
 \hline
 \textbf{TODO} &
 \textbf{Solution} 
\\
 \hline
 \end{tabular} \subsection*{5.2 Symbolic Events}
 \begin{tabular}{|p{.5\textwidth}|p{.5\textwidth}|}
 \hline
 \textbf{TODO} &
 \textbf{Solution} 
\\
 \hline
 \end{tabular} \subsection*{5.3 Symbolic Selection}
 \begin{tabular}{|p{.5\textwidth}|p{.5\textwidth}|}
 \hline
 \textbf{TODO} &
 \textbf{Solution} 
\\
 \hline
 \end{tabular} \subsection*{5.4 SMR Context}
 \begin{tabular}{|p{.5\textwidth}|p{.5\textwidth}|}
 \hline
 \textbf{TODO} &
 \textbf{Solution} 
\\
 \hline
 \end{tabular} \section*{6 multilingual metadata}
 \subsection*{6.1 multilingual metadata}
 \begin{tabular}{|p{.5\textwidth}|p{.5\textwidth}|}
 \hline
 \textbf{TODO} &
 \textbf{Solution} 
\\
 \hline
	evaluate Dublin Core and UNIMARK, check the Java mechanisms for multilingual text &
  
\\ 	\hline
 \end{tabular} \subsection*{6.2 adding MPEG7 metadata}
 \begin{tabular}{|p{.5\textwidth}|p{.5\textwidth}|}
 \hline
 \textbf{TODO} &
 \textbf{Solution} 
\\
 \hline
	research MPEG7 &
  
\\ 	\hline
 \end{tabular} \section*{7 scalable solution}
 \subsection*{7.1 different renderings of the same piece}
 \begin{tabular}{|p{.5\textwidth}|p{.5\textwidth}|}
 \hline
 \textbf{TODO} &
 \textbf{Solution} 
\\
 \hline
 \end{tabular} \subsection*{7.2 piece looks different to different renderes}
 \begin{tabular}{|p{.5\textwidth}|p{.5\textwidth}|}
 \hline
 \textbf{TODO} &
 \textbf{Solution} 
\\
 \hline
 \end{tabular} \section*{8 rendering on different devices}
 \subsection*{8.1 rendering on different devices}
 \begin{tabular}{|p{.5\textwidth}|p{.5\textwidth}|}
 \hline
 \textbf{TODO} &
 \textbf{Solution} 
\\
 \hline
 \end{tabular} \section*{9 measured and unmeasured representation}
 \subsection*{9.1 showing barred and unbarred scores}
 \begin{tabular}{|p{.5\textwidth}|p{.5\textwidth}|}
 \hline
 \textbf{TODO} &
 \textbf{Solution} 
\\
 \hline
 \end{tabular} \subsection*{9.2 showing that no bars are used internally}
 \begin{tabular}{|p{.5\textwidth}|p{.5\textwidth}|}
 \hline
 \textbf{TODO} &
 \textbf{Solution} 
\\
 \hline
 \end{tabular} \section*{10 smybolic elements are uniquely addressable}
 \subsection*{10.1 smybolic elements are uniquely addressable}
 \begin{tabular}{|p{.5\textwidth}|p{.5\textwidth}|}
 \hline
 \textbf{TODO} &
 \textbf{Solution} 
\\
 \hline
 \end{tabular} \section*{11 allowing context to change rendering of elements}
 \subsection*{11.1 allowing context to change rendering of elements}
 \begin{tabular}{|p{.5\textwidth}|p{.5\textwidth}|}
 \hline
 \textbf{TODO} &
 \textbf{Solution} 
\\
 \hline
 \end{tabular} \subsection*{11.2 default contexts}
 \begin{tabular}{|p{.5\textwidth}|p{.5\textwidth}|}
 \hline
 \textbf{TODO} &
 \textbf{Solution} 
\\
 \hline
 \end{tabular} \section*{12 multiple representations of pitch}
 \subsection*{12.1 rendered pitch independent of coded pitch}
 \begin{tabular}{|p{.5\textwidth}|p{.5\textwidth}|}
 \hline
 \textbf{TODO} &
 \textbf{Solution} 
\\
 \hline
 \end{tabular} \subsection*{12.2 tuning of pitch without changing the coded piece}
 \begin{tabular}{|p{.5\textwidth}|p{.5\textwidth}|}
 \hline
 \textbf{TODO} &
 \textbf{Solution} 
\\
 \hline
 \end{tabular} \section*{13 temp and tempo variations}
 \subsection*{13.1 allow tempo variation}
 \begin{tabular}{|p{.5\textwidth}|p{.5\textwidth}|}
 \hline
 \textbf{TODO} &
 \textbf{Solution} 
\\
 \hline
 \end{tabular} \section*{14 rendering from pure logical information}
 \subsection*{14.1 rendering from pure logical information}
 \begin{tabular}{|p{.5\textwidth}|p{.5\textwidth}|}
 \hline
 \textbf{TODO} &
 \textbf{Solution} 
\\
 \hline
 \end{tabular} \section*{15 linear browsing}
 \subsection*{15.1 page markers}
 \begin{tabular}{|p{.5\textwidth}|p{.5\textwidth}|}
 \hline
 \textbf{TODO} &
 \textbf{Solution} 
\\
 \hline
	direct access to symbolic selection??? &
  
\\ 	\hline
 \end{tabular} \section*{16 audio from pure logical information}
 \subsection*{16.1 audio from pure logical information}
 \begin{tabular}{|p{.5\textwidth}|p{.5\textwidth}|}
 \hline
 \textbf{TODO} &
 \textbf{Solution} 
\\
 \hline
 \end{tabular} \section*{17 acknowledge symbolic qualifiers for rendering}
 \subsection*{17.1 acknowledge symbolic qualifiers for rendering}
 \begin{tabular}{|p{.5\textwidth}|p{.5\textwidth}|}
 \hline
 \textbf{TODO} &
 \textbf{Solution} 
\\
 \hline
 \end{tabular} \subsection*{17.2 allow user defined semantics for symbolic qualifiers}
 \begin{tabular}{|p{.5\textwidth}|p{.5\textwidth}|}
 \hline
 \textbf{TODO} &
 \textbf{Solution} 
\\
 \hline
 \end{tabular} \subsection*{17.3 use default values if user defined semantics are missing}
 \begin{tabular}{|p{.5\textwidth}|p{.5\textwidth}|}
 \hline
 \textbf{TODO} &
 \textbf{Solution} 
\\
 \hline
 \end{tabular} \section*{26 multilingual lyrics}
 \subsection*{26.1 multilingual lyrics}
 \begin{tabular}{|p{.5\textwidth}|p{.5\textwidth}|}
 \hline
 \textbf{TODO} &
 \textbf{Solution} 
\\
 \hline
	see multilingual metadata &
  
\\ 	\hline
 \end{tabular} \section*{27 view only parts of the score/partitura}
 \subsection*{27.1 view only parts of the score/partitura}
 \begin{tabular}{|p{.5\textwidth}|p{.5\textwidth}|}
 \hline
 \textbf{TODO} &
 \textbf{Solution} 
\\
 \hline
 \end{tabular} \section*{28 several symbolic selections}
 \subsection*{28.1 symbolic selections for formatting information}
 \begin{tabular}{|p{.5\textwidth}|p{.5\textwidth}|}
 \hline
 \textbf{TODO} &
 \textbf{Solution} 
\\
 \hline
 \end{tabular} \subsection*{28.2 symbolic selections for navigation}
 \begin{tabular}{|p{.5\textwidth}|p{.5\textwidth}|}
 \hline
 \textbf{TODO} &
 \textbf{Solution} 
\\
 \hline
 \end{tabular} \subsection*{28.3 using a selection to display information}
 \begin{tabular}{|p{.5\textwidth}|p{.5\textwidth}|}
 \hline
 \textbf{TODO} &
 \textbf{Solution} 
\\
 \hline
 \end{tabular} \section*{29 allow different scales and temperaments}
 \subsection*{29.1 different scale}
 \begin{tabular}{|p{.5\textwidth}|p{.5\textwidth}|}
 \hline
 \textbf{TODO} &
 \textbf{Solution} 
\\
 \hline
 \end{tabular} \subsection*{29.2 different temperaments}
 \begin{tabular}{|p{.5\textwidth}|p{.5\textwidth}|}
 \hline
 \textbf{TODO} &
 \textbf{Solution} 
\\
 \hline
 \end{tabular} \section*{30 transposing full or partly scores}
 \subsection*{30.1 transposing full scores}
 \begin{tabular}{|p{.5\textwidth}|p{.5\textwidth}|}
 \hline
 \textbf{TODO} &
 \textbf{Solution} 
\\
 \hline
 \end{tabular} \subsection*{30.2 transposing parts with multiple voices}
 \begin{tabular}{|p{.5\textwidth}|p{.5\textwidth}|}
 \hline
 \textbf{TODO} &
 \textbf{Solution} 
\\
 \hline
 \end{tabular} \subsection*{30.3 transposing of segments with multiple voices}
 \begin{tabular}{|p{.5\textwidth}|p{.5\textwidth}|}
 \hline
 \textbf{TODO} &
 \textbf{Solution} 
\\
 \hline
 \end{tabular} \section*{31 rendering hints for symbolic elements}
 \subsection*{31.1 take rendering hints into account}
 \begin{tabular}{|p{.5\textwidth}|p{.5\textwidth}|}
 \hline
 \textbf{TODO} &
 \textbf{Solution} 
\\
 \hline
 \end{tabular} \subsection*{31.2 allow rendering hints at different levels}
 \begin{tabular}{|p{.5\textwidth}|p{.5\textwidth}|}
 \hline
 \textbf{TODO} &
 \textbf{Solution} 
\\
 \hline
 \end{tabular} \section*{32 user defined annotations}
 \subsection*{32.1 annotations for selections}
 \begin{tabular}{|p{.5\textwidth}|p{.5\textwidth}|}
 \hline
 \textbf{TODO} &
 \textbf{Solution} 
\\
 \hline
 \end{tabular} \subsection*{32.2 annotations with events}
 \begin{tabular}{|p{.5\textwidth}|p{.5\textwidth}|}
 \hline
 \textbf{TODO} &
 \textbf{Solution} 
\\
 \hline
 \end{tabular} \subsection*{32.3 annotations with qualifiers}
 \begin{tabular}{|p{.5\textwidth}|p{.5\textwidth}|}
 \hline
 \textbf{TODO} &
 \textbf{Solution} 
\\
 \hline
 \end{tabular} \subsection*{32.4 annotations with context}
 \begin{tabular}{|p{.5\textwidth}|p{.5\textwidth}|}
 \hline
 \textbf{TODO} &
 \textbf{Solution} 
\\
 \hline
 \end{tabular} \end{document}